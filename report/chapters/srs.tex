

This Software Requirements Specification (SRS) outlines the software and system requirements for EthnoVerse. It details functional and non-functional requirements for system capabilities. The document also includes external interfaces to clarify the system’s design and interactions.

\section{Functional Requirements}

This section specifies the functional requirements of the system. The requirements are divided into two primary categories: those pertaining to the \textbf{web platform} and those related to the \textbf{3D virtual tour module} developed using Neural Radiance Fields (NeRF). Each subsection includes a tabular summary and a module-wise functional outline for clarity.

\subsection{EthnoVerse Web Platform}

The web platform is responsible for storing, displaying, and managing multimedia data related to community archives. Only administrators have access to content management features, while public users can freely view and search content without authentication.


% \paragraph{Module-wise Outline}

\begin{outline}
  \1 \textbf{Module 1: Admin Authentication}
    \2 Secure login and session management.
    \2 Only authenticated admins can perform CRUD operations on content.

  \1 \textbf{Module 2: Archive Content Management}
    \2 Upload of multimedia items with validated metadata.
    \2 Edit or delete entries as required, with logs maintained for accountability.
    \2 Metadata management supporting version control and validation.

  \1 \textbf{Module 3: Search and Discovery}
    \2 Keyword search across titles, tags, and descriptions.
    \2 Filtering by community, content type, or upload date.

  \1 \textbf{Module 4: Content Presentation}
    \2 Responsive user interface (React.js) displaying media grid and detailed views.
    \2 Embedded media players for image, audio, and video files.

  \1 \textbf{Module 5: System Administration}
    \2 Dashboard summarizing total uploads and user activity.
    \2 Automated data backups and integrity verification.
    \2 Logs for uploads, edits, deletions, and admin sessions.
    
  \1 \textbf{Module 6: System Extensibility and Scalability}
    \2 Supports addition of new communities and datasets without structural changes.
    \2 Allows integration of new media types through modular schema design.
    \2 Ensures smooth performance as data volume and content grow.


\end{outline}


\begin{table}[H]
\centering
\small
\begin{tabular}{p{1.5cm} p{3cm} p{8cm} p{2cm}}
\hline
\textbf{ID} & \textbf{Function Name} & \textbf{Description} & \textbf{Users} \\
\hline
FR-W01 & Admin Authentication & Admins log in securely using authentication to access upload, edit, and delete functionalities. & Admin \\
FR-W02 & Media Upload & Admin uploads multimedia content (text, image, audio, video) with metadata (title, tags, community, description, consent). & Admin \\
FR-W03 & Edit/Delete Content & Admin edits metadata or removes outdated/incorrect items. All edits are logged. & Admin \\
FR-W04 & Metadata Management & Validates and stores metadata; supports updates and versioning. & Admin \\
FR-W05 & Search Content & Users search items by title, tags, or description. & Admin, Public \\
FR-W06 & Filter Results & Users filter items by community, media type, or upload date. & Admin, Public \\
FR-W07 & View Media & Public users view multimedia content through responsive layouts. & Public \\
FR-W08 & Dashboard Overview & Admin dashboard displays statistics such as total uploads and media distribution. & Admin \\
FR-W09 & Backup and Data Integrity & Automated database backups and consistency checks ensure data reliability. & System \\
FR-W10 & Logs and Audit Trail & Records admin actions (login, upload, edit, delete) for transparency. & System, Admin \\
FR-W11 & Scalability and Generalization & System supports addition of new communities without structural changes or degradation. Database schema and API design ensure scalability and modular extensibility. & System, Admin \\

\hline
\end{tabular}
\caption{Summary of Functional requirements for EthnoVerse web platform.}
\end{table}

\subsection{3D Reconstruction and Virtual Tour}

The 3D module demonstrates the technical application of computer graphics techniques to reconstruct community environments from photographs.

% \paragraph{Module-wise Outline}

\begin{outline}
  \1 \textbf{Module 1: Data Preparation}
    \2 Capture or collect photographs of community sites (courtyards, huts, shrines) from multiple angles.
    \2 Preprocess images (resize, align, normalize lighting) before NeRF input.

  \1 \textbf{Module 2: Scene Reconstruction}
    \2 Implement NeRF or Gaussian Splatting for 3D radiance field estimation.
    \2 Generate scene models stored in compressed formats for web use.

  \1 \textbf{Module 3: Rendering and Visualization}
    \2 Convert reconstructed 3D scenes into Three.js environments.
    \2 Embed virtual tours within the frontend interface.

  \1 \textbf{Module 4: Scene Management and Maintenance}
    \2 Store reconstructed models and associated metadata (scene name, date, contributors).
    \2 Allow admin to update or delete 3D scenes.
    \2 Implement caching and loading optimization for performance.

  \1 \textbf{Module 5: Fallback and Error Handling}
    \2 If reconstruction fails or resources are insufficient, generate a static 360° panoramic viewer.
    \2 Log all reconstruction processes and errors for debugging.
\end{outline}

\begin{table}[H]
\centering
\small
\begin{tabular}{p{1.5cm} p{3cm} p{8cm} p{2cm}}
\hline
\textbf{ID} & \textbf{Function Name} & \textbf{Description} & \textbf{Users} \\
\hline
FR-3D01 & Dataset Preparation & Collect and preprocess multiple-angle photographs of community spaces for NeRF input. & Admin \\
FR-3D02 & Scene Reconstruction & Generate 3D radiance field representations using NeRF or Gaussian Splatting techniques. & System \\
FR-3D03 & Rendering and Visualization & Render 3D scenes into navigable virtual tours integrated into the web interface (via Three.js). & Public, Admin \\
FR-3D04 & Scene Management & Store, update, or remove 3D models and their metadata (scene name, location, source images). & Admin \\
FR-3D05 & Performance Optimization & Optimize rendering quality and load times for web deployment. & System \\
FR-3D06 & Fallback Handling & If NeRF reconstruction fails, fallback to photo-based 360° panoramic tour is generated. & System \\
\hline
\end{tabular}
\caption{Summary of Functional requirements for the 3D reconstruction module.}
\end{table}





\section{Non-Functional Requirements}

The following non-functional requirements ensure the performance, reliability, scalability, and ethical compliance of the EthnoVerse system.

\begin{itemize}

\item \textbf{Performance:}
\begin{itemize}
\item The system must support at least 50 concurrent users with average response times under 2 seconds for typical queries.
\item Search results should render within 1--3 seconds for datasets containing up to 10,000 items.
\item Media files must be optimized for web display using compressed images and adaptive video streaming techniques.
\end{itemize}

\item \textbf{Reliability:}
\begin{itemize}
\item The database must maintain at least 99\% uptime under normal conditions.
\item Failed upload sessions should automatically retry or roll back transactions to preserve data integrity.
\item The system should recover gracefully from unexpected shutdowns or network interruptions.
\end{itemize}

\item \textbf{Security:}
\begin{itemize}
\item All administrator passwords must be stored using salted hashing.
\item JWT-based authentication must be enforced for all protected endpoints.
\item HTTPS must be enabled on deployment; role-based authorization will restrict all CRUD operations to authenticated admins.
\end{itemize}

\item \textbf{Data Integrity and Backup:}
\begin{itemize}
\item Automatic weekly database backups will be maintained for both local and cloud storage.
\item Metadata will use version control to track all edits and revisions.
\item Redundant storage solutions will ensure that no data is lost in case of system failure.
\end{itemize}

\item \textbf{Usability:}
\begin{itemize}
\item The user interface must follow WCAG~2.1 accessibility standards and maintain a responsive design across devices.
\item The upload interface should include clearly labeled fields and drag-and-drop functionality.
\item Search and filter options must remain visible and accessible on all main views.
\end{itemize}

\item \textbf{Compliance and Ethics:}
\begin{itemize}
\item Ethical representation of indigenous content will be ensured with support from an SDP professor and student volunteer specializing in ethnographic research.
\item Administrators must verify consent documentation before publishing any community media.
\item Cultural materials will be presented with appropriate attribution and context.
\end{itemize}


\end{itemize}







\section{External Interfaces}

\subsection{User Interfaces}

This section includes our mockup screens and briefly explains them. The Living Archives platform provides two main user-facing interfaces:

\begin{itemize}
\item \textbf{Public Interface (Web Portal):}
\begin{itemize}
\item Accessible via web browser.
\item Features a home page with search bar, category filters, and featured archives.
\item Each media item has a detail page with description, metadata, and related items.
\item 3D tour viewer embedded within the page for immersive experiences.
\end{itemize}

\item \textbf{Admin Interface:}
\begin{itemize}
\item Accessible only to authorized users.
\item Includes upload form, metadata editor, user management, and backup controls.
\end{itemize}
\end{itemize}

\hrule

\begin{figure}[H]
    \centering
    \includegraphics[width=0.9\textwidth]{1.png} % Homepage - 1.png
    \caption{Homepage: Living Archives Main Landing Screen.}
    \label{fig:screen_home}
\end{figure}

\vspace{0.2cm}
\par
This is the initial entry point, providing an immediate overview of the project's mission as a digital storytelling platform and cultural archive. The page presents the system's core features: a brief mission statement, a featured community (Kolhi), and primary navigation buttons (\texttt{Explore Communities} and \texttt{Search Archives}). Users are led directly to the browse or search functionalities.

\vspace{0.2cm}

\begin{figure}[H]
    \centering
    \includegraphics[width=0.9\textwidth]{2.png} % Explore Communities - 2.png
    \caption{Explore Communities View.}
    \label{fig:screen_explore}
\end{figure}

\vspace{0.2cm}
\par
This view allows users (Public/Researchers) to browse the archive based on documented community groups. The interface displays a card-based list of all communities currently archived (e.g., Kolhi, Example Community). Each card includes the community name, location (Tharparkar), a brief introduction, and a link to \texttt{View Details}, which directs the user to the dedicated Community Detail Page.

\vspace{0.2cm}

\begin{figure}[H]
    \centering
    \includegraphics[width=0.9\textwidth]{4.png} % Search Communities - 3.png
    \caption{Search Communities View.}
    \label{fig:screen_search}
\end{figure}

\vspace{0.2cm}
\par
The Search View enables users to efficiently find specific communities or media based on keywords, tags, or descriptive metadata. This page features a universal search bar across all content. The results are presented in cards, and the underlying logic applies to individual media titles and tags to support deep discovery. Users can refine their search by employing filters based on community, content type, or upload date 

\vspace{0.2cm}

\begin{figure}[H]
    \centering
    \includegraphics[width=0.9\textwidth]{5.png} % Community Detail - 4.png
    \caption{Community Detail Page (Kolhi).}
    \label{fig:screen_community_detail}
\end{figure}

\vspace{0.2cm}
\par
This page serves as the dedicated hub for a specific community, showcasing all available content (multimedia and 3D) for spatial immersion. The header provides context and two primary action buttons: \texttt{Open 3D Tour} (linking to the immersive view) and \texttt{View Media} (linking to individual content). The lower section, \texttt{Media Collection}, displays a grid of multimedia items (Image, Audio, Video). This page essentially bridges the archival content with the immersive virtual tour experience

\vspace{0.2cm}

\begin{figure}[H]
    \centering
    \includegraphics[width=0.9\textwidth]{6.png} % 3D Viewer - 5.png
    \caption{3D Virtual Tour Viewer.}
    \label{fig:screen_3d_tour}
\end{figure}

\vspace{0.2cm}
\par
The viewer provides an immersive, navigable virtual tour experience of the reconstructed community environment. The screen hosts the interactive 3D rendering, generated using NeRF and Gaussian Splatting, which is optimized for web deployment. A supporting section, \texttt{About This Tour}, provides context on the ethnographic and technical basis of the reconstruction. 

\vspace{0.2cm}

\begin{figure}[H]
    \centering
    \includegraphics[width=0.9\textwidth]{7.png} % Media Detail (Image) - 6.png
    \caption{Media Detail Page (Image).}
    \label{fig:screen_media_image}
\end{figure}

\vspace{0.2cm}
\par
This page in particular displays the specific image (e.g., \texttt{Traditional Tattoo}) and indicates its media type (\texttt{IMAGE}) and collection. 

\vspace{0.2cm}

\begin{figure}[H]
    \centering
    \includegraphics[width=0.9\textwidth]{8.png} % Media Detail (Audio) - 7.png
    \caption{Media Detail Page (Audio/Oral History).}
    \label{fig:screen_media_audio}
\end{figure}

\vspace{0.2cm}
\par
The interface presents audio content (oral histories, songs) and makes it searchable and accessible through transcription. The view features an embedded audio player and a dedicated \texttt{Transcript} panel below it. The core function is to allow users to search the transcripts for keywords and potentially jump to corresponding timestamps in the media. 


\vspace{0.2cm}

\begin{figure}[H]
    \centering
    \includegraphics[width=0.9\textwidth]{3.png} % About Page - 8.png
    \caption{About Living Archives Page.}
    \label{fig:screen_about}
\end{figure}

\vspace{0.2cm}
\par
This page provides background and rationale to stakeholders, researchers, and the public. This page details the project's roots as a Kaavish initiative, its technical foundation (MERN stack integration), and its commitment to preserving the heritage of Sindh's communities. Its inclusion ensures the user is informed about the project's scope and technology, aligning with the project's goal of openness.







\subsection{Application Program Interface (API)}

The system is built using a \textbf{RESTful API} architecture (Node.js/Express backend) to manage data flow between the React frontend, the MongoDB database, and the Python-based 3D Reconstruction Engine.

\vspace{0.2cm}
\noindent\textbf{Core API Endpoints}
\begin{itemize}
    \item \textbf{Auth Endpoints:} Facilitate user login and session management (\texttt{/auth/login}, \texttt{/auth/register}) using \textbf{JWT} for secure, role-based access control 
    \item \textbf{Content Management Endpoints:} Restricted to Administrators, enabling CRUD operations:
    \begin{itemize}
        \item \texttt{POST /api/media}: Upload new multimedia content (image, audio, video) with metadata.
        \item \texttt{PUT /api/media/\{id\}}, \texttt{DELETE /api/media/\{id\}}: Edit or remove existing archive entries.
    \end{itemize}
    \item \textbf{Search and Discovery Endpoints:} Publicly accessible read operations:
    \begin{itemize}
        \item \texttt{GET /api/communities}: Retrieve a list of all documented communities.
        \item \texttt{GET /api/media?q=keyword\&community=kolhi}: Search content based on keywords and filter by metadata.
    \end{itemize}
    \item \textbf{3D Scene Endpoints:} Used for the virtual tour module:
    \begin{itemize}
        \item \texttt{POST /api/3d/reconstruct}: (Admin/System) Triggers the NeRF/Gaussian Splatting engine for a new scene reconstruction.
        \item \texttt{GET /api/3d/\{sceneId\}/model}: Retrieves the optimized 3D model data (e.g., JSON, point cloud) for rendering in the client's Three.js viewer.
    \end{itemize}
\end{itemize}


\subsection{Hardware and Communication Interfaces}

\begin{itemize}
\item \textbf{Client-Side:} Standard desktop or laptop computers with modern web browser compatibility.
\item \textbf{Server-Side Infrastructure:} The backend and database (MongoDB) are hosted on a cloud environment, supporting modular scaling.
\item \textbf{Specialized Hardware (3D Reconstruction):} A mid-range GPU (e.g., NVIDIA RTX 3060 or equivalent) is recommended for 3D reconstruction tasks, managed by a dedicated service or cloud instance (Non-Functional Requirement: Performance).
\item \textbf{Communication Protocols:} All data transfer, including API calls, media streaming, and asset retrieval, uses HTTPS to ensure data integrity and security.
\end{itemize}






\section{Use Cases}

\subsection{Use Case Diagram}
The Use Case Diagram visually represents the primary capabilities of the EthnoVerse system and the relationship between the system and its key actors: the Public User/Researcher and the Administrator.

\begin{figure}[H]
    \centering
    \includegraphics[width=0.85\textwidth]{chapters/usecase.png}
    \caption{EthnoVerse High-Level Use Case Diagram.}
    \label{fig:use_case_diagram}
\end{figure}

\textbf{Actors:}
\begin{itemize}
    \item \textbf{Public User / Researcher}: Individuals seeking to consume content, browse, search, view media, and explore immersive 3D tours.
    \item \textbf{Administrator}: Authorized personnel responsible for content management, technical maintenance, and data integrity.
\end{itemize}

\hrule

\subsection{Use Cases by Actor}

\subsubsection{Public User / Researcher Use Cases}
These use cases detail the primary ways users interact with the public-facing interface, aligning with the core goals of exploration and access.

\begin{table}[H]
    \centering
    \small
    \begin{tabular}{p{1.5cm} p{4.0cm} p{8.5cm}}
    \hline
    \textbf{ID} & \textbf{Use Case Name} & \textbf{Description} \\
    \hline
    UC-P01 & Access Homepage & As a \textbf{user}, I want to \textbf{view the homepage} so that I can get an overview of the archive's mission and easily navigate to its main features like Explore and Search. \\
    UC-P02 & Browse Communities & As a \textbf{user}, I want to \textbf{browse a list of all available communities} so that I can discover content by cultural or regional grouping. \\
    UC-P03 & Search Archive & As a \textbf{user}, I want to \textbf{search the entire archive using keywords} so that I can find specific media items based on titles, tags, or descriptions. \\
    UC-P04 & Filter Search Results & As a \textbf{user}, I want to \textbf{filter my search results} so that I can narrow down the list by community, media type, or date to find exactly what I'm looking for. \\
    UC-P05 & View Media Detail & As a \textbf{user}, I want to \textbf{view a detailed page for a media item} so that I can see the content, its full description, metadata, and attribution. \\
    UC-P06 & View Media Transcript & As a \textbf{user}, I want to \textbf{view the transcript for audio or video content} so that I can read the text, improve accessibility, and search for specific parts within the media. \\
    UC-P07 & Explore 3D Tour & As a \textbf{user}, I want to \textbf{launch and navigate a 3D virtual tour} so that I can immersively explore a reconstructed community environment. \\
    \hline
    \end{tabular}
    \caption{Use Cases for Public User / Researcher.}
\end{table}

\subsubsection{Administrator Use Cases}
These use cases focus on the control, maintenance, and archival duties performed by authenticated administrators through the secure Admin Interface.

\begin{table}[H]
    \centering
    \small
    \begin{tabular}{p{1.5cm} p{4.0cm} p{8.5cm}}
    \hline
    \textbf{ID} & \textbf{Use Case Name} & \textbf{Description} \\
    \hline
    UC-A01 & Secure Login & As an \textbf{administrator}, I want to \textbf{securely log in to the system} so that I can access the protected content management features. \\
    UC-A02 & Upload New Content & As an \textbf{administrator}, I want to \textbf{upload new media files and their metadata} so that I can add new cultural documentation to the archive. \\
    UC-A03 & Edit Content Metadata & As an \textbf{administrator}, I want to \textbf{edit the metadata of existing items} so that I can correct errors or update information to ensure data accuracy. \\
    UC-A04 & Delete Archived Item & As an \textbf{administrator}, I want to \textbf{delete outdated or non-compliant content} so that I can maintain the ethical and factual integrity of the public archive. \\
    \hline
    \end{tabular}
    \caption{Use Cases for Administrator.}
\end{table}




\section{System Diagram}

% \subsection{System Block Diagram and Description}s

\begin{figure}[h!]
    \centering
    \includegraphics[width=0.95\linewidth]{chapters/systemdiag.png}
    \caption{EthnoVerse System Block Diagram}
    \label{fig:systemblock}
\end{figure}

\noindent
The EthnoVerse architecture follows a modular, service-oriented design built on the MERN stack with a dedicated 3D reconstruction module. Figure~\ref{fig:systemblock} illustrates the high-level components and data flow between them.

\subsubsection*{1. Frontend (React.js)}
The frontend provides a dynamic, responsive user interface that allows users to browse communities, view multimedia archives, and launch immersive 3D tours. It communicates with the backend through RESTful APIs over HTTPS. Administrators access a separate dashboard for content management, uploads, and system monitoring.

\subsubsection*{2. Backend (Node.js + Express)}
The backend serves as the core logic layer responsible for handling API requests, authentication, routing, and data processing. It manages CRUD operations for multimedia content and interacts with both the database and the 3D reconstruction module. It implements secure JWT-based authentication and role-based access control for administrators.

\subsubsection*{3. Database (MongoDB)}
MongoDB stores all structured metadata, user profiles, and references to media files. The schema supports flexible, community-specific expansion and version-controlled updates. It also logs administrative actions and maintains backup copies for recovery and integrity checks.

\subsubsection*{4. Media Storage Service}
All uploaded media—images, audio, video, and 3D assets—are stored in a cloud-based repository. Metadata in MongoDB links each file to its respective community and media type. The storage layer supports redundancy and retrieval optimization for faster access and rendering.

\subsubsection*{5. 3D Reconstruction Engine (NeRF / Gaussian Splatting)}
This module operates as an independent service responsible for generating immersive 3D scenes from multi-angle photographs. Once processed, the resulting models are optimized for web use and integrated into the platform’s frontend via Three.js. The engine can fall back to 360° panoramic tours if GPU or data constraints arise.

\subsubsection*{6. Admin Console}
The Admin Console provides authenticated users with control over system content and maintenance. It enables secure uploads, metadata edits, deletion of outdated items, and performance monitoring. This module also schedules automated database backups and generates logs for transparency.

\subsubsection*{7. Data Flow Summary}
\begin{itemize}
    \item Users interact with the system through the React.js interface.
    \item The frontend communicates with the Node.js backend via RESTful API calls.
    \item The backend retrieves or updates records in MongoDB and media storage.
    \item For 3D content, the backend triggers the reconstruction engine and retrieves the processed model.
    \item The processed 3D scenes and media are delivered back to the frontend for visualization.
\end{itemize}
